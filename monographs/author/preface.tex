%%%%%%%%%%%%%%%%%%%%%%preface.tex%%%%%%%%%%%%%%%%%%%%%%%%%%%%%%%%%%%%%%%%%
% sample preface
%
% Use this file as a template for your own input.
%
%%%%%%%%%%%%%%%%%%%%%%%% Springer %%%%%%%%%%%%%%%%%%%%%%%%%%
% \Extrachap{Preface}
\preface

%% Please write your preface here
Functions of bounded variation ($\mathbf{BV}$ functions in the sequel) have had an important role in several classical problems of the
calculus of variations, for instance in the theory of graphs with minimal area. More recently, this class of functions has been the natural
tool to study several problems characterized by the appearance of discontinuity hypersurfaces; examples come from image segmentation theory 
and fracture mechanics. The analysis of these problems require a knowledge of some of the basic concepts of geometric measure theory, such 
as Hausdorff measures and rectifiable sets.\\
One of the motivations which led us to write this book is the desire to provide a systematic and self-contained presentation of the theory 
of functions of bounded variation and, at the same time, an elementary introduction to geometric measure theory. In fact, after the classical 
treatises of V. G. Mazj'a \cite{Maz_ja_1985}, A. I. Vol'pert and S. I. Hudjaev \cite{Volpert1985AnalysisIC} and H. Federer \cite{Federer_1996} 
(in the latter $\mathbf{BV}$ functions are presented in the more general framework of the currents), some aspects of the theory of $\mathbf{BV}$ 
functions have been treated in the monographs of E. Giusti \cite{Giusti_1984}, U. Massari and M.Miranda \cite{208}, W. Ziemer \cite{278}, 
L.C. Evans and R. F. Gariepy \cite{145}, M.Giaquinta, G. Modica and J. Soucek \cite{173}, but the analysis of fine properties of $\mathbf{BV}$ 
functions and the development of general variational problems in $\mathbf{BV}$ is not the central goal of any of these monographs. The first 
half of our book is, instead, explicitly devoted to the theory of $\mathbf{BV}$ functions, from classical results up to the developments 
of the last ten years.\\
\indent Our starting point is, in Chapter 1, abstract measure theory. We assume the reader has an elementary knowledge of the subject, and 
we emphasize some aspects perhaps less widely known, but fundamental for the development of the book, such as weak convergence in spaces 
of measures, outer measures and Caratheodory construction.\\
In the second chapter we introduce all the basic ingredients of geometric measure theory, such as Hausdorff measures $\mathcal{H}^k$, 
covering theorems, rectifiable sets, area and coarea formulae, Minkowski content. Moreover, the chapter contains a brief treatment of Young 
measures and of the continuity and semicontinuity properties of functionals defined on measures. The aim is to give a quite general 
presentation, without restricting e.g. to the case of hypersurfaces, which is the only one relevant for the development of the $\mathbf{BV}$ 
theory. In our treatment of geometric measure theory a fundamental role is played by Lipschitz functions: indeed, these functions are more 
flexible than $\mathit{C}^1$ functions with respect, for instance, to truncation and extension and, by the classical Rademacher theorem,
they are almost everywhere differentiable. Hence, as shown by H. Federer in \cite{152}, the canonical liberalization techniques 
can be adapted to this context. In particular , we develop the whole theory without using the link between Lipschitz and $\mathit{C}^1$ 
functions provided by the Whitney extension theorem. Another feature of the chapter and of the subsequent one is the emphasis on the 
so-called blow-up technique, which is used both for the study of the local properties of rectifiable sets and for the fine theory of $\mathbf{BV}$ 
functions. In this respect, a unifying concept is that if tangent measure, introduced (adapting with minor variants the original definition of
D. Preiss in \cite{242}) in Section ~\ref{section 2.7}  


\begin{equation}
  J(\Gamma, u)\coloneqq \int_{R\backslash \Gamma}^{}\|\nabla u\|^2 +\alpha\|u-g\|^2 dx + \beta \mathcal{H}^{N-1} (R\cap \Gamma)
\end{equation}



\vspace{\baselineskip}

\noindent \textit{Pisa} \hfill L. A.\\
\textit{Florence} \hfill N. F.\\
\textit{Lecce} \hfill D. P.\\
June 1999\\
% \begin{flushright}\noindent
% Berlin, \today \hfill {\it Ali Darijani}\\
% \end{flushright}
