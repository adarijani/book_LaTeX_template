%%%%%%%%%%%%%%%%%%%%%%foreword.tex%%%%%%%%%%%%%%%%%%%%%%%%%%%%%%%%%
% sample foreword
%
% Use this file as a template for your own input.
%
%%%%%%%%%%%%%%%%%%%%%%%% Springer %%%%%%%%%%%%%%%%%%%%%%%%%%

\foreword
%% Please have the foreword written here
Surely, Antoni Zygmund's \textit{Trigonometric Series} has been, and continues to be, one of the most influential 
books in the history of mathematical analysis.Therefore, the current printing, which ensures the future availability of 
this work to the mathematical public is an event of major importance. Its tremendous longevity is a testimony 
to its depth and clarity. Generations of mathematicians from Hardy and Littlewood to recent classes of graduate 
students specializing in analysis have viewed \textit{Trigonometric Series} with enormous admiration and have 
profited greatly from reading it. In light of the importance of Antoni Zygmund as a mathematician and of the 
impact of \textit{Trigonometric Series}, it is only fitting that a brief discussion of his life and mathematics accompany 
the present volume, and this is what I have attempted to give here\footnote[1]{I have been fortunate to have a number of 
excelent references to consult regarding the life of Antoni Zygmund. The reader interested in aditional material 
may consult the references in the bibliography to this Foreword.}. I can only hope that it provides at least 
a small glimpse into the story of this masterpiece and of the man who produced it.\\
\indent Antoni Zygmund was born on December 26, 1900 in Warsaw, Poland. His parents had received relatively 
little education, and were of modest means, so his background was far less privileged than that of the vast majority 
of his colleagues. Zygmund attended school through the middle of high school in Warsaw. When World War I broke 
out, his family was evacuated to Poltava in the Ukraine, where he continued his studies. When the war ended 
in 1918, his family returned to Warsaw, where he completed pre collegiate work, and entered Warsaw University. 
Zygmund's main interest throughout his childhood was astronomy, but at Warsaw University at that time there 
were not sufficient courses offered in that subject to make it realistic as a specialization, and so 
Zygmund turned instead toward another of his interests, mathematics.\\
\indent  There were a number of excelent mathematicians and teachers who profoundly influenced Zygmund during 
this period. The greatest impact came from Aleksander Rajchman and Stanislaw Saks. Rajchman was a junior 
faculty member who was an expert on Riemann's theory of trigonometric series, and Saks a felow student 
who was a few years his senior. From Rajchman, he learned much of the Riemann theory, and his doctoral thesis 
in 1923 was on this subject. Zygmund became an active colaborator with both Rajchman and Saks, publishing 
a number of important articles with each of them. In adition, Saks and Zygmund produced \textit{Analytic Functions}, 
one of the classic texts on complex analysis.\\
\indent One year prior to his PhD, Zygmund was appointed to an instructorship at the Warsaw Polytechnical 
School, and, in 1926, he was appointed Privatdozent at the University of Warsaw. He was awarded a Rockefeller fellowship, 
which he used to travel to England for the academic year of 1929-30 and visit G.H. Hardy at Cambridge 
for the first half of the year, and J.E. Litlewod at Oxford for the second half. This experience had a tremendous 
impact on the young Zygmund. Not only did he work with two of the greatest analysts of the time, but while in 
England, he also met another young mathematician, R.E.A.C. Paley, a student of Littlewod, with whom he had 
an extended and very fruitful collaboration. When he returned to Poland in 1930, Zygmund moved to Wilno where 
he took a chair in mathematics at the Stefan Batory University. It was here that Zygmund's talent and quiet charismasa 
as a teacher of advanced mathematics began to have a very major impact on the field. In the fall of 1930, 
Zygmund met a new student at the University, Jozef Marcinkiewicz. Marcinkiewicz was recognized, even when he 
was a student, as being tremendously talented, with the potential to become a serious mathematician. In the 
following year, which was only the second at Stefan Batory for both teacher and student, Zygmund decided to 
offer a course on trigonometric series preceded by lectures on Lebesgue integration. Marcinkiewicz attended 
this course, and thus began his association with Zygmund. It took just three years for Marcinkiewicz to obtain 
his masters degree, with a thesis that contained the highly non-trivial result that it is posible for 
a continuous periodic function to have interpolating polynomials corresponding to equidistant nodal points 
diverging almost everywhere. This result was elaborated to form his PhD thesis in 1935, and in 1937 Marcinkicwicz 
became Dozent in Wilno. In the period from 1935 to 1939, a collaboration between Marcinkiewicz and Zygmund 
developed that was incredibly successful. Though of relatively short duration, their work opened a number of 
new directions, and in a sense set the stage for the theory of singular integrals which would be Antoni Zygmund's 
greatest contribution.\\
\indent 






\vspace{\baselineskip}
\begin{flushright}\noindent
University of Chicago, \today \hfill {\it Robert A. Fefferman}\\
\end{flushright}


