%%%%%%%%%%%%%%%%%%%%%%foreword.tex%%%%%%%%%%%%%%%%%%%%%%%%%%%%%%%%%
% sample foreword
%
% Use this file as a template for your own input.
%
%%%%%%%%%%%%%%%%%%%%%%%% Springer %%%%%%%%%%%%%%%%%%%%%%%%%%
\Extrachap{Foreward}
%% Please have the foreword written here
Surely, Antoni Zygmund's \textit{Trigonometric Series} has been, and continues to be, one of the most influential 
books in the history of mathematical analysis.Therefore, the current printing, which ensures the future availability of 
this work to the mathematical public is an event of major importance. Its tremendous longevity is a testimony 
to its depth and clarity. Generations of mathematicians from Hardy and Littlewood to recent classes of graduate 
students specializing in analysis have viewed \textit{Trigonometric Series} with enormous admiration and have 
profited greatly from reading it. In light of the importance of Antoni Zygmund as a mathematician and of the 
impact of \textit{Trigonometric Series}, it is only fitting that a brief discussion of his life and mathematics accompany 
the present volume, and this is what I have attempted to give here\footnote[1]{I have been fortunate to have a number of 
excelent references to consult regarding the life of Antoni Zygmund. The reader interested in aditional material 
may consult the references in the bibliography to this Foreword.}. I can only hope that it provides at least 
a small glimpse into the story of this masterpiece and of the man who produced it.\\
\indent Antoni Zygmund was born on December 26, 1900 in Warsaw, Poland. His parents had received relatively 
little education, and were of modest means, so his background was far less privileged than that of the vast majority 
of his colleagues. Zygmund attended school through the middle of high school in Warsaw. When World War I broke 
out, his family was evacuated to Poltava in the Ukraine, where he continued his studies. When the war ended 
in 1918, his family returned to Warsaw, where he completed pre collegiate work, and entered Warsaw University. 
Zygmund's main interest throughout his childhood was astronomy, but at Warsaw University at that time there 
were not sufficient courses offered in that subject to make it realistic as a specialization, and so 
Zygmund turned instead toward another of his interests, mathematics.\\
\indent  There were a number of excelent mathematicians and teachers who profoundly influenced Zygmund during 
this period. The greatest impact came from Aleksander Rajchman and Stanislaw Saks. Rajchman was a junior 
faculty member who was an expert on Riemann's theory of trigonometric series, and Saks a felow student 
who was a few years his senior. From Rajchman, he learned much of the Riemann theory, and his doctoral thesis 
in 1923 was on this subject. Zygmund became an active colaborator with both Rajchman and Saks, publishing 
a number of important articles with each of them. In adition, Saks and Zygmund produced \textit{Analytic Functions}, 
one of the classic texts on complex analysis.\\
\indent One year prior to his PhD, Zygmund was appointed to an instructorship at the Warsaw Polytechnical 
School, and, in 1926, he was appointed Privatdozent at the University of Warsaw. He was awarded a Rockefeller fellowship, 
which he used to travel to England for the academic year of 1929-30 and visit G.H. Hardy at Cambridge 
for the first half of the year, and J.E. Litlewod at Oxford for the second half. This experience had a tremendous 
impact on the young Zygmund. Not only did he work with two of the greatest analysts of the time, but while in 
England, he also met another young mathematician, R.E.A.C. Paley, a student of Littlewod, with whom he had 
an extended and very fruitful collaboration. When he returned to Poland in 1930, Zygmund moved to Wilno where 
he took a chair in mathematics at the Stefan Batory University. It was here that Zygmund's talent and quiet charismasa 
as a teacher of advanced mathematics began to have a very major impact on the field. In the fall of 1930, 
Zygmund met a new student at the University, Jozef Marcinkiewicz. Marcinkiewicz was recognized, even when he 
was a student, as being tremendously talented, with the potential to become a serious mathematician. In the 
following year, which was only the second at Stefan Batory for both teacher and student, Zygmund decided to 
offer a course on trigonometric series preceded by lectures on Lebesgue integration. Marcinkiewicz attended 
this course, and thus began his association with Zygmund. It took just three years for Marcinkiewicz to obtain 
his masters degree, with a thesis that contained the highly non-trivial result that it is possible for 
a continuous periodic function to have interpolating polynomials corresponding to equidistant nodal points 
diverging almost everywhere. This result was elaborated to form his PhD thesis in 1935, and in 1937 Marcinkicwicz 
became Dozent in Wilno. In the period from 1935 to 1939, a collaboration between Marcinkiewicz and Zygmund 
developed that was incredibly successful. Though of relatively short duration, their work opened a number of 
new directions, and in a sense set the stage for the theory of singular integrals which would be Antoni Zygmund's 
greatest contribution.\\
\indent The years in which Zygmund was a young profesor in Wilno, though very productive mathematically, 
were not easy ones. This was due in large part to Zygmund's courageous opposition to the bigotry which was 
all to common around him, and which was supported by the higher authorities. An example of this was his 
strong dislike of anti-Semitic policies with in his university. At one time,for instance, student organizations, 
somewhat analogous to modern day fraternities, were sufficiently influential to mandate that all Jewish 
students must sit on the left side of each classroom during lectures. For Zygmund, this was completely unacceptable 
and in response, he decided to move his classes from the larger hals to small mathematics department seminar rooms 
where there were only long tables in a central arrangement, and hence no seats at the left or right of the 
room. Another illustration of Zygmund's sensitivity to issues of social justice had to do with is university's 
requirement that all student associations have faculty members as their academic sponsors. Zygmund 
regularly sponsored associations which were not in favor with the Polish government. These unpopular moves on 
Zygmund's part did not go unnoticed, and in the year 1931, as part of the political purges of the universities by 
the government, Zygmund was dismissed from his professorship. This immediately brought extremely strong reaction 
from some of the most distinguished mathematicians in Europe.From Lebesgue in France, and  from Hardy and 
Littlewod in England came formal written protests which resulted in Zygmund's reinstatement as professor. 
It is therefore an important aspect of Zygmund's life that, in a very real sense, he was a crusader for human 
rights well before this was fashionable.\\
\indent Among the many remarkable contributions of the Wilno period is the writing of the first version of 
this book, published in Warsaw under the title \textit{Trigonometrical Series}. This was Zygmund's first book, 
and it was published as volume \romannumeral 14 of the series Monografie Matematyczne. This is the same series in which the 
celebrated book \textit{Theorie des Operations Lineaires} by S. Banach appears as volume .The tremendous success 
of \textit{Trigonometrical Series} led to its expansion and revision in to a second edition, published in 1959 
by Cambridge University Press, and then to no fewer than six reprinted versions after that.\\
\indent The time in Wilno which featured the rapid achievement of success came to a sudden end in September 1939 
as World War 2 erupted. At that time, both Zygmund and Marcinkicwicz were mobilized as reserve oficers in 
the Polish army, and, as a result of the temporary ``friendship" between Germany and Russia, Poland was partitioned. 
The Soviets were given control of much of the country, including the part containing Wilno, and they preceded to round up 
and execute many of the Polish officer corps in the Katyn Forest massacre. Most likely, this is how Marcinkiewicz 
perished. Almost by a miracle, Zygmund was able to return to his family and escape with them to the United States, 
but his loss was absolutely devastating. His principal collaborators up to that time besides Marcinkicwicz had 
been Saks, Rajchmanand, Paley. Both Saks and Rajchman were murdered by the Nazis, and Paley had died in a 
tragic accident in 1933. These loses were not just mathematical. Zygmund had been extremely close to each of 
them, and so the war period must surely have been one of the most difficult of his life.\\
\indent By 1939, Zygmund had an international reputation, and many friends all over the mathematical world. It was due 
to the efforts of some of these friends, such as Jacob Tamarkin, Jerzy Neyman and Norbert Wiener, that Zygmund was 
able to emigrate to the United States in 1940. During the time immediately prior to the United States entering 
in to the war, there were very few jobs available to mathematicians. Nevertheless, after teaching for a 
semester at MIT, Zygmund was offered and accepted a position at Mount Holyoke College in central Massachusetts. 
A few years later, other offers followed. In 1945, Zygmund became a professor at the University of Pennsylvania, 
and then, in 1947, he was offered a professorship at the University of Chicago.\\
\indent The University of Chicago mathematics department, which had had a tradition of great strength, 
had experienced a period of decline prior to World War 2. During the war, the president of the university, 
Robert Maynard Hutchins, brought the Manhatan project to the campus, and with it camea number of outstanding 
scientists, such as Enrico Fermi. Hutchins then decided to make it a priority to strengthen the mathematics 
department in order to match the high quality of physical science apointments that had been made. To this end, 
a new chairman, Marshal Stone, was brought to the university and asked to bring about this improvement. 
The result was some thing phenomenal. In the period just after the war, Stone was able to asemble one of the 
best mathematics departments in history. At this time, the faculty of mathematics included such members as A.A. Albert, 
S.S. Chern, L. Graves, P. Halmos, I. Kaplansky, S. MacLane, I. Segal, E. Spanier, M. Stone, A. Weil and A. Zygmund. 
Together with this influx of great mathematicians there came a corresponding influx of brilliant students.\\
\indent The combination of such a strong mathematician and teacher as Zygmund with the unusualy rich mathematical 
environment of the University of Chicago produced a golden period of creativity and of supervision of exceptional 
students for Zygmund that was the crowning achievement of his life's work. In several cases, the route of 
outstanding students to Chicago was not totally straightforward, and the most famous case was that of Alberto P. Calderon. 
The story of the means by which Calderon came to Chicago is legendary. The following, taken from the introduction 
to the book, \textit{Essays in Honor of Alberto P.Calderon}~\cite{albertohonor} tells the story beautifuly:\\
\noindent In the years imediately after World War 2, the U.S. Department of State had a very active visitors 
program that sent prominent scientists to Latin America. Thus, Adrian Albert, Marshal Stone, and George Birkhoff 
visited Buenos Aires, and Gonzalez Dominguez arranged through them the visit of Zygmund, whose work on Fourier Series 
he much admired. At the Institute of Mathematics, Zygmund gave a two-month seminar on topics in analysis, 
based on his book. This seminar was atended by Gonzalez Dominguez, Calderon, Mischa Cotlar, and three other 
young Argentine mathematicians. Each of the participants had to discus a portion of the text. Calderon's 
assignment was to present the Marcel Riesz theorem on the continuity of the Hilbert transform in $L^p$. 
According to Cotlar's vivid recollection of the event, Calderon's exposition was entirely aceptable to the 
junior audience, but not to Zygmuncl, who appeared agitated and grimaced all the time. Finally, he interrupted 
Calderon abruptly to ask where had read the material he was presenting, and a bewildered Calderon answered 
that he had read it in Zygmund's book. Zygmund vehemently informed the audience that this was not the proof 
in his book, and after the lecture took Calderon aside and quized him about the new short and elegant proof. 
Calderon confessed that he had first tried to prove the theorem by himself, and then thinking he could not 
do it, had read the begining of the proof in the book; but after the first couple of lines, instead of turning 
the page, had figured out how the proof would finish. In fact, he had found himself an elegant new proof of 
the Riesz Theorem! Zygmund imediately recognized Calderon's power and then and there decided to invite him to 
Chicago to study wit him.\\
\indent This anecdote illustrates one of Calderon's main characteristics\dots \\ \\
\noindent The anecdote above also illustrates one of Zygmund's main characteristics: His tremendous desire to 
work with people of the greatest mathematic ability, and his absolute devotion to those people. Calderon 
came to the University of Chicago in 1949 on a Rockefeller fellowship, and only one year later received his 
PhD there under Zygmund's supervision. The thesis consisted of three research papers, each of which was a major 
work. In particular, among the results of the thesis was one of the greatest importance, concerning the boundary 
behavior of harmonic functions of several variables, which represented a crucial step in carying out the real 
variable program of Zygmund which will be described below. The collaboration betwen Calderon and Zygmund which 
followed was certainly one of the greatest in the history of modern analysis, and created a theory, the so-called 
Calderon-Zygmund Theory of Singular Integrals, that not only allowed for the extension of much of classical 
Fourier analysis from one to several dimensions,but played a fundamental role in the development of the theories 
of partial differential equations and geometry as well.\\
\indent More than simply creating a new powerful mathematical theory at Chicago, Zygmund created a school, 
the Chicago School of Analysis, which was to have an enormous impact on the subject in the next five decades, 
and promises to continue to do so in the future. After Calderon, there came other students who worked with 
Zygmund and who individually made historic contributions to mathematics. In 1955, Elias M. Stein received his 
doctorate under Zygmund, and, as is well known, by his brilliant research and teaching went on to establish 
a great school at Princeton. A bit later, other remarkable students finished their thesis work with Zygmund, 
including Paul Cohen and Guido and Mary Weiss. Taking into account the generations of students whose mathematical 
ancestry is traceable back to Zygmund, it is hard to imagine what mathematical analysis would be like without 
their collective contribution.\\
\indent At Chicago, Zygmund had a total of thirty-five students. His collected works include some 215 articles. 
Zygmund received many formal honors in his lifetime. He was a recipient of the Steele Prize of the American 
Mathematical Society, as well as the National Medal of Science, the highest award given by the United States 
government in recognition of scientific achievement. In addition, he was given membership of a number of academics, 
including the National Academy of Sciences and the American Academy for Arts and Sciences(USA), the Polish 
Academy of Sciences, the Argentina Academy of Sciences, Royal Academy of Sciences of Spain, and the Academy, 
of Arts and Sciences of Palermo, Italy. Zygmund also held honorary degrees from Washington University, the 
University of Torun, Poland, the University of Paris and the University of Uppsala, Sweden.\\
\indent After a very long and productive life in which he published his last, research article at the age of 
79, he finally slowed considerably, and, after a long illness, died at the age of 91. Few mathematicians have 
provided such a striking and wonderful counter example to G.H. Hardy's view on the rapidity of loss of creativity 
that mathematicians suffer with age.\\
\indent Zygmund's life events and his mathematics, particularly that covered in the present volume, are heavily 
intertwined. In what follows, I would like to discuss this mathematics in the context of the historical perspective 
considered above.\\
\indent continued:-)






\vspace{\baselineskip}
\begin{flushright}\noindent
University of Chicago, \hfill {\it Robert A. Fefferman}\\
\end{flushright}


